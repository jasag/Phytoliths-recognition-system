\apendice{Planificación}

\section{Introducción}
Para llevar a cabo este proyecto vamos a aplicar una metodología llamada Scrum. Scrum es una metodología de desarrollo software ágil, es decir, durante cada \textit{sprint} \footnote{\textit{Sprint}: es el período en el cual se lleva a cabo el trabajo en sí.\cite{wiki:scrum}}, generalmente cada semana, se asignarán unas determinadas tareas a cumplimentar, con un producto como consecuencia de estas tareas. Al final de cada \textit{sprint} se realizará una reunión junto a los tutores para validar los avances realizados y determinar las tareas a realizar durante el siguiente \textit{sprint}.

\section{Planificación temporal}

En esta sección podremos ver la planificación del proyecto subdividida en \textit{sprints}, como previamente comentaba. En cada uno de los sprints se detalla las tareas a realizar, algunos detalles descriptivos y un gráfico \textit{burndown}.

% \section{Estudio previo}

\subsection{\textit{Sprint} 0}
Estas son las tareas a realizar durante este \textit{sprint} 0:

\begin{itemize}
	\item Probar \LaTeX.
	\item Gestor de tareas/versiones: \textit{Github} y \textit{Zenhub}.
	\item Instalar \textit{Anaconda} y \textit{Jupyter}.
	\item Leer los artículos propuestos por los tutores.
	\item Comenzar a probar algunos algoritmos de binarización.
\end{itemize}

Como se puede ver las tareas a realizar son básicas, puesto que es el \textit{sprint} 0 y es un \textit{sprint} de mera adaptación al entorno de trabajo. La única tarea que supone un esfuerzo de comprensión mayor es la lectura de los artículos propuestos sobre trabajos relacionados o con una problemática similar a la nuestra. A continuación, en la figura \ref{fig:A.1.1}, se muestra el diagrama \textit{burndown} de este \textit{sprint}. 

\begin{figure}[h]
\centering
\includegraphics[width=0.99\textwidth]{sprint_0}
\caption{Burndown del \textit{sprint} 0}
\label{fig:A.1.1}
\end{figure}

\subsection{\textit{Sprint} 1}
Estas son las tareas a realizar durante esta \textit{sprint} 1:

\begin{itemize}
	\item Documentar lo realizado durante el \textit{sprint} 0.
	\item Documentar lo que se irá realizando durante este \textit{sprint} 1.
	\item Continuar probando con algoritmos de procesamiento de imágenes.
	\item Probar una aproximación con clasificadores al problema.
\end{itemize}

Puesto que en el \textit{sprint} anterior no se documentó lo realizado, durante este se pretende documentar todo lo realizado durante el \textit{sprint} anterior y este. Además de continuar probando con algoritmos de procesamiento de imágenes y comenzar a probar con la aproximación al problema mediante clasificadores.

En este \textit{sprint} me vi desbordado de trabajo debido a la subestimación del esfuerzo a empeñar en las distintas tareas. No siendo capaz de comenzar a probar una aproximación con clasificadores. Por ello la tarea <<Probar una aproximación con clasificadores al problema>> se vio movida al siguiente \textit{sprint}. 

A continuación, en la figura \ref{fig:A.1.2}, se muestra el diagrama \textit{burndown} de este \textit{sprint}. El cual tiene dicho aspecto debido a que muchas de las tareas se trabajaron de manera paralela, no siendo acabadas hasta el final del \textit{sprint}. Y, además, algunas de las tareas no fueron cerradas cuando se debió, aspecto que se corregirá en los siguientes \textit{sprints}.

\begin{figure}
\centering
\includegraphics[width=0.99\textwidth]{sprint_1}
\caption{Burndown del \textit{sprint} 1}
\label{fig:A.1.2}
\end{figure}

\subsection{\textit{Sprint} 2}
Estas son las tareas a realizar durante este \textit{sprint} 2:

\begin{itemize}
	\item Probar una aproximación con clasificadores al problema.
	\item Aplicación del método "Non maximum suppression" sobre el clasificado.
\end{itemize}

Puesto que la aproximación mediante reconocimiento de imágenes no reflejaba unos resultados muy positivos, durante la reunión mantenida con los tutores se decidió el uso de una técnica distinta. Nos referimos a la utilización  de un clasificador, junto a un descriptor visual.

Debido a que todavía no se poseían suficientes imágenes para el estudio del problema mediante esta técnica, lo que se decidió es aplicarla sobre otro problema de características similares, como es el reconocimiento de caras en imágenes. Con unos resultados bastante positivos debido a distintos razonamientos explicados en la Memoria, sección de Aspectos relevantes del proyecto.

A continuación, en la figura \ref{fig:A.1.3}, se muestra el diagrama \textit{burndown} de este \textit{sprint}.

\begin{figure}
\centering
\includegraphics[width=0.99\textwidth]{sprint_2}
\caption{Burndown del \textit{sprint} 2}
\label{fig:A.1.3}
\end{figure}

\subsection{\textit{Sprint} 3}
Estas son las tareas a realizar durante este \textit{sprint} 3:

\begin{itemize}
	\item Reorganizar los \textit{Jupyter Notebooks}.
	\item Probar distintos clasificadores y métricas.
	\item Enviar fotos rotadas al clasificador.
\end{itemize}

Durante este \textit{sprint}, primero, se reorganizo la estructura del proyecto. Aportando mucho más orden y claridad a nuestro proyecto. Después, se introdujeron múltiples clasificadores y métricas, los cuales introduciré en mayor medida en la memoria, como \textit{Random Forest} o \textit{Gradient tree boosting}. Por último, se enviaron imágenes rotadas al clasificador, con el fin de poder analizar una posible problemática.

A continuación, en la figura \ref{fig:A.1.4}, se muestra el diagrama \textit{burndown} de este \textit{sprint}.

\begin{figure}
\centering
\includegraphics[width=0.99\textwidth]{sprint_3}
\caption{Burndown del \textit{sprint} 3}
\label{fig:A.1.4}
\end{figure}

\subsection{\textit{Sprint} 4}
Estas son las tareas a realizar durante este \textit{sprint} 4:

\begin{itemize}
	\item Implementación de \textit{Data Augmentation} en nuestro conjunto de entrenamiento.
	\item Implementación de controles de usuario.
\end{itemize}

Durante este \textit{sprint} se aplicó en nuestro conjunto de entrenamiento la técnica \textit{Data augmentation}. Esta técnica nos permitió aumentar el tamaño de nuestro conjunto de entrenamiento enormemente. 

Además, se realizó un \textit{notebook}\footnote{Siempre que nos referimos a un \textit{notebook}, a lo que nos referimos es a un \textit{Jupyter Notebook}}, con controles de usuario, los cuales nos permiten escoger entre clasificadores, imágenes y probabilidades. Permitiendo la continua interacción entre el usuario y la clasificación de una imagen, sin la necesidad de modificar el código por parte del usuario del notebook para cambiar entre las distintas opciones.

A continuación, en la figura \ref{fig:A.1.5}, se muestra el diagrama \textit{burndown} de este \textit{sprint}.

\begin{figure}
\centering
\includegraphics[width=0.99\textwidth]{sprint_4}
\caption{Burndown del \textit{sprint} 4}
\label{fig:A.1.5}
\end{figure}


\subsection{\textit{Sprint} 5}
Estas son las tareas a realizar durante este \textit{sprint} 5:

\begin{itemize}
	\item Implementar un \textit{file chooser} 
	\item Añadir más clasificadores.
	\item Correciones en la documentación.
	\item Estudiar como implementar un etiquetador de imágenes.
\end{itemize}

Durante este \textit{sprint} se añadieron los clasificadores que deseábamos, es decir, un clasificador bayesiano y un clasificador mediante regresión logística. Además, se añadió un \textit{file chooser} que nos permitiría, desde ese momento, escoger la imagen que deseemos dentro de nuestro sistema operativo. En cuanto a la documentación, se corrigió toda la  realizada hasta ese momento. Y, por último, se hizo un estudio básico sobre como implementar un etiquetador de imágenes mediante un \textit{Widget} de \textit{Python}. Aunque, esta última tarea no tuviese ningún producto resultante en este \textit{sprint}.

A continuación, en la figura \ref{fig:A.1.6}, se muestra el diagrama \textit{burndown} de este \textit{sprint}.

\begin{figure}
\centering
\includegraphics[width=0.99\textwidth]{sprint_5}
\caption{Burndown del \textit{sprint} 5}
\label{fig:A.1.6}
\end{figure}

\subsection{\textit{Sprint} 6}
Estas son las tareas a realizar durante este \textit{sprint} 6:

\begin{itemize}
	\item Estudiar los \textit{Widgets} personalizados de \textit{Jupyter Notebook} e \textit{Ipython}.
\end{itemize}

Aunque este \textit{sprint} se encuentre compuesto por una única tarea, no era menos complejo por ello. El objetivo de este \textit{sprint} era obtener un \textit{Widget} capaz de etiquetar imágenes. Pero en la realización de este se encontraron multiples problemas. Obteniendo como producto resultante tres posibles alternativas con aspectos a corregir.

Por lo tanto, en la figura \ref{fig:A.1.7} mostramos el diagrama \textit{burndown}, poco esclarecedor, de este \textit{sprint}.

\begin{figure}
\centering
\includegraphics[width=0.99\textwidth]{sprint_6}
\caption{Burndown del \textit{sprint} 6}
\label{fig:A.1.7}
\end{figure}



\subsection{\textit{Sprint} 7}
Estas son las tareas a realizar durante este \textit{sprint} 7:

\begin{itemize}
	\item Estudiar \textit{Bag of Words}.
	\item Añadir mayor parametrización al \textit{Jupyter Notebook UI}.
	\item Corregir \textit{bugs} del Widget previamente implementado.
\end{itemize}

Durante este \textit{sprint} se consiguió, en primer lugar, corregir una de las alternativas del etiquetador de imágenes, o \textit{Widget}, desarrolladas durante el \textit{sprint} anterior. Además, se corrigieron y añadieron múltiples parámetros en el \textit{Jupyter Notebook UI} y en las clases utilizadas por este \textit{Notebook}. Y, por ultimo, se realizo un estudio sobre un modelo ampliamente usado para tareas de clasificación, llamado \textit{Bag of Words}.

A continuación, en la figura \ref{fig:A.1.8}, se muestra el diagrama \textit{burndown} de este \textit{sprint}.

\begin{figure}
\centering
\includegraphics[width=0.99\textwidth]{sprint_7}
\caption{Burndown del \textit{sprint} 7}
\label{fig:A.1.8}
\end{figure}


\subsection{\textit{Sprint} 8}
Estas son las tareas a realizar durante este \textit{sprint} 8:

\begin{itemize}
	\item Implementar la funcionalidad de obtención de imágenes en el etiquetador de imágenes, o \textit{Widget}.
	\item Mejorar la interfaz del etiquetador de imágenes.
	\item Crear un primer prototipo de interfaz de usuario.
\end{itemize}

Durante este \textit{sprint} se realizó un primer prototipo de interfaz de  usuario. Partiendo de este prototipo, se mejoró la interfaz del etiquetador de imágenes. Consiguiendo, así, una interfaz adecuada para el cliente. Además, se implemento la funcionalidad que nos permitiría obtener una imágen resultante de cada etiqueta realizada en las distintas imágenes.

A continuación, en la figura \ref{fig:A.1.9}, se muestra el diagrama \textit{burndown} de este \textit{sprint}.

\begin{figure}
\centering
\includegraphics[width=0.99\textwidth]{sprint_8}
\caption{Burndown del \textit{sprint} 8}
\label{fig:A.1.9}
\end{figure}

\subsection{\textit{Sprint} 9}

Este \textit{sprint} tendrá una duración de dos semanas. Debido  a la carga de trabajo asociada a este \textit{sprint} y al ser días no lectivos por las vacaciones de Semana Santa. 

Estas son las tareas a realizar durante este \textit{sprint} 9:

\begin{itemize}
	\item Añadir un texto a cada etiqueta que realizamos en una imagen.
	\item Añadir notificaciones al usuario en la carga y guardado de imágenes.
	\item Guardar las coordenadas de las etiquetas de cada imagen.
	\item Cargar las etiquetas de una imagen que haya sido previamente etiquetada.
	\item Controlar que el usuario no cree etiquetas en el SVG pero fuera de la imagen.
	\item Añadir la posibilidad de eliminar etiquetas previamente realizadas.
	\item Corregir los notebooks creados para la técnica Bag of Words.
\end{itemize}


\begin{comment}
\begin{figure}
\centering
\includegraphics[width=0.99\textwidth]{sprint_9}
\caption{Burndown del \textit{sprint} 9}
\label{fig:A.1.10}
\end{figure}
\end{comment}


\section{Estudio de viabilidad}

\subsection{Viabilidad económica}
En esta sección se realiza un análisis de los costes económicos que hubiera supuesto el desarrollo de este proyecto en un entorno empresarial.

\subsubsection{Coste del personal}
Este proyecto ha sido desarrollado por un único desarrollador a tiempo parcial. En la tabla \ref{tabla:costespersonales} muestro el desglose de los costes ocasionados por el salario que hubiese recibido en una situación real.

\tablaSmallSinColores{Costes de personal}{p{4.5cm} p{.25cm} p{2.5cm}}{costespersonales}{
  \multicolumn{3}{p{6cm}}{\textbf{Costes de personal}} \\
 }
 {
  Salario mensual neto  & 1500\euro{} \\
  Retención IRPF (15\%) & \euro{} \\
  Seguridad social (29,9\%) & \euro{} \\
  Salario mensual bruto  & \euro{} \\\hline
  Salario total(4 meses)  & \euro{} \\
  }

\subsubsection{Coste del \textit{software} y \textit{hardware}}



\subsubsection{Costes totales}



\subsection{Viabilidad legal}

En este apartado enunciare las distintas librerías utilizadas junto a sus licencias, en todos sus casos de código abierto. Véase la tabla \ref{tabla:licencias}. Si se desea indagar más sobre las distintas librerías se recomienda ver la sección de herramientas y técnicas dentro de la memoria.

\tablaSmallSinColores{Licencias de las librerías}{p{3.5cm} p{1.5cm} p{2.5cm}}{licencias}{
  \multicolumn{3}{p{7cm}}{\textbf{Licencias de las librerías}} \\
 }
 {
  Librería & Versión & Licencia \\\hline
  Numpy & 1.12 & BSD \\
  scikit-learn & 0.18 & BSD \\
  scikit-image & 0.13 & BSD \\
  Matplotlib & 1.12 & PSF \\
  Jupyter Notebook & 1 & BSD \\
  Jupyter Dashboards & 0.6 & BSD \\
  Ipython File Upload & 0.1.2 & MIT \\
  darkflow & 0 & GPL 3 \\
 }
 
 Finalmente, este proyecto esta publicado bajo la licencia \textit{BSD 3-clause} con la que se permite un libre uso, modificación, distribución y uso privado de este. Sin embargo, con la condición de que el código debe ser suministrado en todas las ocasiones junto a la licencia que expone las distintas garantías de uso. Para finalizar, esta licencia introduce una muy limitada responsabilidad sobre la utilización de este proyecto y ningún tipo de garantía.