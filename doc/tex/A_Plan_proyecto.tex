\apendice{Planificación}

\section{Introducción}
En las primeras semanas, de manera previa a utilizar cualquier metodología de desarrollo software, debemos de probar distintas aproximaciones a nuestro problema. Para que finalmente podamos escoger la más adecuada entre las distintas posibilidades.

Aunque no se aplique una metodología ágil, se asignarán unas tareas semanales para llevar a cabo la traceabilidad del desarrollo del proyecto, así como el control del estudio sobre la problemática a realizar durante cada semana o \textit{milestone}.

Una vez escogida la mejor solución posible comenzaremos a utilizar una metodología ágil de desarrollo.

\section{Estudio previo}

\subsection{Semana 0}
Estas son las tareas a realizar durante esta semana 0:

\begin{itemize}
	\item Probar LaTeX
	\item Gestor de tareas/versiones: Github y Zenhub
	\item Instalar anaconda y Jupyter
	\item Leer los artículos propuestos por los tutores
	\item Comenzar a probar algunos algoritmos de binarización
\end{itemize}

Como se puede ver las tareas a realizar son básicas puesto que es la semana 0 y es una semana de mera adaptación al entorno de trabajo. La única tarea que supone un esfuerzo de comprensión mayor es la lectura de los artículos propuestos sobre trabajos relacionados o con una problemática similar a la nuestra. A continuación en la figura \ref{fig:A.1.1} se muestra el diagrama \textit{burndown} de esta semana. 

\begin{figure}[h]
\centering
\includegraphics[width=0.99\textwidth]{semana_0}
\caption{Burndown de la semana 0}
\label{fig:A.1.1}
\end{figure}

\subsection{Semana 1}
Estas son las tareas a realizar durante esta semana 1:

\begin{itemize}
	\item Documentar lo realizado durante la semana 0
	\item Documentar lo que se irá realizando durante esta semana 1
	\item Continuar probando con algoritmos de procesamiento de imágenes
	\item Probar una aproximación con clasificadores al problema
\end{itemize}

Puesto que en la semana anterior no se documento lo realizado, durante esta semana se pretende documentar todo lo realizado durante la semana anterior y esta semana. Además de continuar probando con algoritmos de procesamiento de imágenes y comenzar a probar con la aproximación al problema mediante clasificadores.

En esta semana me vi desbordado de trabajo debido a la subestimación del esfuerzo a empeñar en las distintas tareas. No siendo capaz de comenzar a probar una aproximación con clasificadores. Por ello la tarea "Probar una aproximación con clasificadores al problema" se vio movida a la siguiente semana. 

A continuación en la figura \ref{fig:A.1.2} se muestra el diagrama \textit{burndown} de esta semana.

\begin{figure}[h]
\centering
\includegraphics[width=0.99\textwidth]{semana_1}
\caption{Burndown de la semana 1}
\label{fig:A.1.2}
\end{figure}

\subsection{Semana 2}
Estas son las tareas a realizar durante esta semana 2:

\begin{itemize}
	\item Probar una aproximación con clasificadores al problema
	\item Aplicación del método "Non maximum suppression" sobre el clasificado
\end{itemize}

Puesto que la aproximación mediante reconocimiento de imágenes no reflejaba unos resultados muy positivos, durante la reunión mantenida con los tutores se decidió el uso de una técnica distinta. Está técnica es mediante una SVM, utilizado como clasificador, junto a diversas técnicas.

Debido a que todavía no se poseían suficientes imágenes para el estudio del problema directamente mediante esta técnica, lo que se decidió es aplicar estas técnicas sobre otro problema de características similares, como es el reconomiento de caras en imágenes. Con unos resultados bastante positivos debido a distintos razonamientos explicados en la sección de Aspectos relevantes del proyecto.

\begin{comment}
\begin{figure}[h]
\centering
\includegraphics[width=0.99\textwidth]{semana_1}
\caption{Burndown de la semana 1}
\label{fig:A.1.1}
\end{figure}
\end{comment}

\section{Planificación temporal}

\section{Estudio de viabilidad}

\subsection{Viabilidad económica}

\subsection{Viabilidad legal}


