\capitulo{7}{Conclusiones y Líneas de trabajo futuras}

%Todo proyecto debe incluir las conclusiones que se derivan de su desarrollo. Éstas pueden ser de diferente índole, dependiendo de la tipología del proyecto, pero normalmente van a estar presentes un conjunto de conclusiones relacionadas con los resultados del proyecto y un conjunto de conclusiones técnicas. 
%Además, resulta muy útil realizar un informe crítico indicando cómo se puede mejorar el proyecto, o cómo se puede continuar trabajando en la línea del proyecto realizado. 

\section{Conclusiones}

\section{Lineas futuras de trabajo}

Los fitolitos poseen una complejidad enorme por la multitud de formas de estos, incluso perteneciendo a un mismo tipo, y la multitud de tipos de fitolitos existentes. Por lo tanto, existe un gran margen de mejora en el reconocimiento automático de estos.

El problema fundamental, junto al ya expuesto sobre la complejidad de las formas de los fitolitos, es la inexistencia de un conjunto de imágenes etiquetadas de fitolitos. Actualmente, las técnicas más avanzadas para el reconocimiento de objetos necesitan de conjuntos de entrenamiento muy grandes. Por ello, en futuros desarrollos se podrían mejorar las técnicas de \textit{data augmentation} e incorporar nuevas técnicas con el objetivo de obtener un conjunto de entrenamiento sustancialmente mayor.

Una de las nuevas técnicas a las que me refiero es el aprendizaje mediante modelos 3D, el cual nos permitiría crear un conjunto de imágenes significativamente mayor~\cite{sem}~\cite{3dmodels}. Estas técnicas consiguen recrear un modelo 3D de un objeto a partir de diferentes imágenes. Y a partir del modelo generado obtener imágenes desde las distintas perspectivas.

Otra posible mejora sería la realización del entrenamiento de \textit{YOLO} en un entorno con mayores recursos, sobre todo con la utilización de una mejor tarjeta gráfica, como la \textit{Nvidia Titan X}. O incluso con la utilización de varias tarjetas gráficas concurréntemente, lo cual aceleraría el proceso de entrenamiento significativamente.