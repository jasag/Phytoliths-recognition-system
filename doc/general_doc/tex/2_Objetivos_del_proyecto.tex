\capitulo{2}{Objetivos del proyecto}

%Se puede distinguir entre los objetivos marcados por los requisitos del software a construir y los objetivos de carácter técnico que plantea a la hora de llevar a la práctica el proyecto.

En este apartado se explican los distintos objetivos identificados en este proyecto. Distinguiendo entre los objetivos generales del proyecto y los objetivos técnicos.

\section{Objetivos generales}

Los objetivos generales que plantea este proyecto son los siguientes:

\begin{itemize}
	\item Realizar un estudio de las técnicas del estado del arte que solucionen el reconocimiento automático de fitolitos con la mejor precisión y eficiencia posible.
	\item Crear una aplicación para el etiquetado de fitolitos, mediante la cual se extraiga toda la información necesaria.
	\item Crear un sistema de reconocimiento automático de fitolitos. Mediante el cual, un usuario sea capaz de introducir cualquier imagen que desee para realizar este reconocimiento automático.
\end{itemize}

\section{Objetivos técnicos}

Los objetivos técnicos que plantea este proyecto son los siguientes:

\begin{itemize}
	\item Utilizar \textit{Python} para crear todo lo que involucra este sistema, como lenguaje de programación principal.
	\item Usar librerias para \textit{Python}, como \textit{scikit-image}\cite{scikit-image}, que nos permitan llevar a cabo las tareas más complejas del proyecto.
	\item Crear una aplicación, basada en los \textit{Jupyter Notebooks}, que permita una fácil instalación y uso a los usuarios.
	\item Utilizar un sistema de control de versiones, en nuestro caso \textit{Git}, junto a un servicio central, \textit{GitHub}. Para mantener un correcto control de los productos generados.
	\item Utilizar alguna herramienta para la gestión del proyecto, en nuestro caso \textit{ZenHub}. Que nos permita hacer un seguimiento de la metodología Scrum.
	\item Utilizar herramientas de prototipado para llevar a cabo las interfaces de usuario.
\end{itemize}
