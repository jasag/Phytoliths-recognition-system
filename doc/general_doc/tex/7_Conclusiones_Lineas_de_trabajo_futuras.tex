\capitulo{7}{Conclusiones y Líneas de trabajo futuras}

%Todo proyecto debe incluir las conclusiones que se derivan de su desarrollo. Éstas pueden ser de diferente índole, dependiendo de la tipología del proyecto, pero normalmente van a estar presentes un conjunto de conclusiones relacionadas con los resultados del proyecto y un conjunto de conclusiones técnicas. 
%Además, resulta muy útil realizar un informe crítico indicando cómo se puede mejorar el proyecto, o cómo se puede continuar trabajando en la línea del proyecto realizado. 

\section{Conclusiones}

Desde el inicio de este proyecto se nos han presentado multitud de problemáticas. Por hacer resumen, la más importante y que más nos ha condicionado en el desarrollo del proyecto es que inicialmente no teníamos imágenes etiquetadas, lo cual como he reiterado en más de una ocasión es fundamental para un problema de este tipo. Esto nos llevó a desarrollar un etiquetador. Una vez terminado este, nuestros clientes y colaboradores del CSIC nos suministraron un conjunto de unas 160 imágenes. Las cuales no eran suficientes directamente pero que mediante técnicas de \textit{data augmentation} supimos solucionar.

Llegados a este punto, principalmente tratamos de aplicar dos soluciones: aprendizaje profundo y ventana deslizante. Tras más de un mes tratando de entrenar a \textit{YOLO}, nuestro modelo, no fuimos capaces de obtener predicciones minimamente correctas. Por lo que pasamos a la segunda opción, la cual resulto siendo nuestra solución final.

Concluyendo, leyendo estos dos últimos párrafos y, sobre todo, lo expuesto a lo largo de todo este trabajo hemos conseguido solventar múltiples problemáticas. Y, además, establecer las lineas base de trabajo para sistemas más avanzados.

En cuanto a lo que ha supuesto para mi, este trabajo me ha mejorado en muchos aspectos tanto técnicos como profesionales. Refiriéndome a los técnicos ha supuesto en que adquiera conocimientos sobre procesamiento de imágenes, inteligencia artificial, \textit{Python}, librerias de \textit{Python}, \textit{JavaScript}, entre otros. En cuanto a los personales, me ha supuesto un reto personal al tener que compatibilizarlo con las prácticas curriculares y por el esfuerzo intrínseco de un trabajo de este tipo. Por lo que, a pesar de que existen múltiples aspectos mejorables, me encuentro muy satisfecho con el trabajo realizado en todos los aspectos.

\section{Lineas futuras de trabajo}

Los fitolitos poseen una complejidad enorme por la multitud de formas de estos, incluso perteneciendo a un mismo tipo, y la multitud de tipos de fitolitos existentes. Por lo tanto, existe un gran margen de mejora en el reconocimiento automático de estos.

El problema fundamental, junto al ya expuesto sobre la complejidad de las formas de los fitolitos, es la inexistencia de un conjunto de imágenes etiquetadas de fitolitos. Actualmente, las técnicas más avanzadas para el reconocimiento de objetos necesitan de conjuntos de entrenamiento muy grandes. Por ello, en futuros desarrollos se podrían mejorar las técnicas de \textit{data augmentation} e incorporar nuevas técnicas con el objetivo de obtener un conjunto de entrenamiento sustancialmente mayor.

Una de las nuevas técnicas a las que me refiero es el aprendizaje mediante modelos 3D, el cual nos permitiría crear un conjunto de imágenes significativamente mayor~\cite{sem,3dmodels}. Estas técnicas consiguen recrear un modelo 3D de un objeto a partir de diferentes imágenes. Y a partir del modelo generado obtener imágenes desde las distintas perspectivas.

Otra posible mejora sería la realización del entrenamiento de \textit{YOLO} en un entorno con mayores recursos, sobre todo con la utilización de una mejor tarjeta gráfica, como la \textit{Nvidia Titan X}. O incluso con la utilización de varias tarjetas gráficas concurréntemente, lo cual aceleraría el proceso de entrenamiento significativamente.